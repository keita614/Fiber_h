\documentclass[dvipdfmx]{jlreq}
\usepackage{amsmath,amsfonts, amssymb}
\usepackage{bm}
\usepackage{physics}
\usepackage[dvipdfmx]{graphicx}
\usepackage{tikz} 


\newcommand{\psicone}{\psi_{\mathrm{c1}}}
\newcommand{\psictwo}{\psi_{\mathrm{c2}}}
\newcommand{\psimin}{\psi_{\mathrm{min}}}
\newcommand{\psimax}{\psi_{\mathrm{max}}}
\newcommand{\thetamin}{\theta_{\mathrm{min}}}
\newcommand{\thetamax}{\theta_{\mathrm{max}}}
\newcommand{\phimin}{\phi_{\mathrm{min}}}
\newcommand{\phimax}{\phi_{\mathrm{max}}}
\newcommand{\Latt}{L_{\mathrm{att}}}
\newcommand{\mumin}{\mu_{\mathrm{min}}}

\title{光ファイバー内で考えられる光の伝播モデル}
\date{\today}
\author{湯淺 圭太 \\ 大阪公立大学 理学研究科 宇宙線物理学研究室}
\begin{document}
\maketitle
\newpage
\tableofcontents
\newpage

\section{はじめに}
当研究室では、IceCube-Gen2で考えられるveto能力の低下を防ぐために、波長変換機能を持った光ファイバーを用いた集光器の開発を行っている。
この集光器において、どれほどの光を集光できるかを評価するために、光ファイバーそのものの性能評価を行う必要がある。
そのためのシミュレーションとして、Geant4を用いた光ファイバー内での光の伝播シミュレーションを行っている。
このGeant4シミュレーション以外の手法を用いて、光ファイバー内での光の伝播を評価することができれば、シミュレーション結果の妥当性を確認することができる。
そこで、本稿では光ファイバー内での光の伝播モデルを構築し、光ファイバー内での光の伝播について解析的に評価することを試みる。

\section{光ファイバー内での光子の伝播モデル}

\section{コア軸からの距離$a$の関数としてのTrapping Efficiency}

\section{ファイバー端からの距離$z$でのTrapping Efficiency}

\section{ファイバー内での光子のさまざまなパラメータの分布}

\section{Fiber.hに実装されている関数の仕様}

\end{document}