\documentclass[dvipdfmx]{jlreq}
\usepackage{amsmath,amsfonts, amssymb}
\usepackage{bm}
\usepackage{physics}
\usepackage[dvipdfmx]{graphicx}
\usepackage{tikz} 


\newcommand{\psicone}{\psi_{\mathrm{c1}}}
\newcommand{\psictwo}{\psi_{\mathrm{c2}}}
\newcommand{\psimin}{\psi_{\mathrm{min}}}
\newcommand{\psimax}{\psi_{\mathrm{max}}}
\newcommand{\thetamin}{\theta_{\mathrm{min}}}
\newcommand{\thetamax}{\theta_{\mathrm{max}}}
\newcommand{\phimin}{\phi_{\mathrm{min}}}
\newcommand{\phimax}{\phi_{\mathrm{max}}}
\newcommand{\Latt}{L_{\mathrm{att}}}
\newcommand{\mumin}{\mu_{\mathrm{min}}}

\title{光ファイバー内で考えられる光の伝播モデル}
\date{\today}
\author{湯淺 圭太 \\ 大阪公立大学 理学研究科 宇宙線物理学研究室}
\begin{document}
\maketitle
\newpage
\tableofcontents
\newpage

\section{はじめに}
当研究室では、IceCube-Gen2で考えられるveto能力の低下を防ぐために、波長変換機能を持った光ファイバーを用いた集光器の開発を行っている。
この集光器において、どれほどの光を集光できるかを評価するために、光ファイバーそのものの性能評価を行う必要がある。
そのためのシミュレーションとして、Geant4を用いた光ファイバー内での光の伝播シミュレーションを行っている。
このGeant4シミュレーション以外の手法を用いて、光ファイバー内での光の伝播を評価することができれば、シミュレーション結果の妥当性を確認することができる。
そこで、本稿では光ファイバー内での光の伝播モデルを構築し、光ファイバー内での光の伝播について解析的に評価することを試みる。

\section{光ファイバー内での光子の伝播モデル}
光ファイバーの中で光線は$\bm{x}_0 = (a, 0, 0)$で発光し、方向ベクトル$\bm{v}$で伝播し、半径$d$の円筒の$\bm{x}_1 = (x_1, y_1, z_1)$ でファイバー表面に衝突する(Figure \ref{fig:Single_cladding})。
光線の式は以下のように表される。
\begin{align}
  \bm{x} &= t \bm{v} + \bm{x}_0 = t \begin{pmatrix}
    \sin \theta \cos \phi \\
    \sin \theta \sin \phi \\
    \cos \theta
  \end{pmatrix} + \begin{pmatrix}
    a \\
    0 \\ 
    0
  \end{pmatrix}.
\end{align}
$x_1^2 + y_1^2 = d^2$を用いると、
\begin{align}
  x_1^2 + y_1^2 = (t \sin \theta \cos \phi + a)^2 + (t \sin \theta \sin \phi)^2 &= d^2 \nonumber \\
  (t \sin \theta)^2 + 2 a \cos \phi (t \sin \theta) + a^2 - d^2 &= 0 \nonumber \\
  t \sin \theta = -a \cos \phi + \sqrt{a^2 \cos^2 \phi - a^2 + d^2} &= -a \cos \phi + \sqrt{d^2 - a^2 \sin^2 \phi} \nonumber \\
  &\equiv -a \cos \phi + b, \; \; \;  b \equiv \sqrt{d^2 - a^2 \sin^2 \phi}. \label{eq:solt}
\end{align}
したがって、ファイバー表面に衝突する点$\bm{x}_1$の座標$(x_1, y_1)$は以下のように表される。
\begin{align}
  x_1 &= t \sin \theta \cos \phi + a = -a \cos^2 \phi + b \cos \phi  + a \label{eq:normx}\\
  y_1 &= t \sin \theta \sin \phi = -a \cos \phi \sin \phi + b \sin \phi. \label{eq:normy} 
\end{align}
$\bm{x}_1 = (x_1, y_1, z_1)$におけるファイバー表面での単位法線ベクトルは$(x_1, y_1, 0)$である。
\eqref{eq:normx}、\eqref{eq:normy}および$x_1^2 + y_1^2 = d^2$を満たすことから、$\bm{x}$での単位法線ベクトルの成分は$\bm{n}_1 = (x_1/d, y_1/d, 0)$となる。
\begin{figure}[h]
  \centering
  \includegraphics[width=0.9\columnwidth]{Figure/Single_cladding_fiber.png}
  \caption{Single-cladding fiber}
  \label{fig:Single_cladding}
\end{figure}

光のファイバー表面への入射角$\psi_1$(法線ベクトル$\bm{n}_1$に対する角度)は$\cos \psi_1 = \bm{n}_1 \cdot \bm{v}$を満たし、$\cos \psi_1$は次のように計算される。
\begin{align}
  \cos \psi_1 &= \frac{1}{d} \left( -a \sin \theta \cos^3 \phi + b \sin \theta \cos^2 \phi + a \sin \theta \cos \phi - a \sin \theta \cos \phi \sin^2 \phi + b \sin \theta \sin^2 \phi \right) \nonumber \\
  &= \frac{1}{d} \left( a \sin \theta \cos \phi (-\cos^2 \phi + 1 - \sin^2 \phi) + b \sin \theta \right) = \frac{b}{d} \sin \theta = \sin \theta \sqrt{1 - (a/d)^2 \sin^2 \phi}
 \label{eq:cospsi}
\end{align}
内部での全反射は、スネルの法則より入射角$\psi_1$が以下の関係を満たす時に起こる。
\begin{align}
  \sin \psi_1 &\ge \sin \psicone \equiv \frac{n_{c1}}{n_f} \label{eq:criticalangle1}
\end{align}
ここで、$\psicone$はファイバーのコアの屈折率$n_f$とクラッドの屈折率$n_{c1}$から決まる臨界角である。
全反射を起こす時に取り得る$\theta$の最大値$\theta_{\mathrm{max}}$は、$\psicone$と次のように対応する。
\begin{align}
  \cos \psicone &= \frac{b}{d} \sin \thetamax(\phi)  = \sin \thetamax(\phi) \sqrt{1 - (a/d)^2 \sin^2 \phi}.
\end{align}

光ファイバー内で光子が連続して2回反射する間の移動距離の差は次のように表される。
\begin{align}
\Delta \ell &= \left| t_+ - t_- \right| = \frac{2}{\sin \theta} b = \frac{2}{\sin \theta} \sqrt{d^2 - a^2 \sin^2 \phi}. \label{eq:deltaell} 
\end{align}

\section{コア軸からの距離$a$の関数としてのTrapping Efficiency}

\section{ファイバー端からの距離$z$でのTrapping Efficiency}

\section{ファイバー内での光子のさまざまなパラメータの分布}

\section{Fiber.hに実装されている関数の仕様}

\end{document}