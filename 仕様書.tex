\documentclass[dvipdfmx]{jlreq}
\usepackage{amsmath,amsfonts, amssymb}
\usepackage{bm}
\usepackage{physics}
\usepackage[dvipdfmx]{graphicx}
\usepackage{tikz} 


\newcommand{\psicone}{\psi_{\mathrm{c1}}}
\newcommand{\psictwo}{\psi_{\mathrm{c2}}}
\newcommand{\psimin}{\psi_{\mathrm{min}}}
\newcommand{\psimax}{\psi_{\mathrm{max}}}
\newcommand{\thetamin}{\theta_{\mathrm{min}}}
\newcommand{\thetamax}{\theta_{\mathrm{max}}}
\newcommand{\phimin}{\phi_{\mathrm{min}}}
\newcommand{\phimax}{\phi_{\mathrm{max}}}
\newcommand{\Latt}{L_{\mathrm{att}}}
\newcommand{\mumin}{\mu_{\mathrm{min}}}

\title{光ファイバー内で考えられる光の伝播モデル}
\date{\today}
\author{湯淺 圭太 \\ 大阪公立大学 理学研究科 宇宙線物理学研究室}
\begin{document}
\maketitle
\newpage
\tableofcontents
\newpage

\section{はじめに}
当研究室では、IceCube-Gen2で考えられるveto能力の低下を防ぐために、波長変換機能を持つ光ファイバーを用いた集光器の開発を行っている。
この集光器において、どれほどの光を集光できるかを評価するために、光ファイバーそのものの性能評価を行う必要がある。
そのためのシミュレーションとして、Geant4を用いた光ファイバー内での光の伝播シミュレーションを行っている。
このGeant4シミュレーション以外の手法を用いて、光ファイバー内での光の伝播を評価することができれば、Geant4のシミュレーション結果の妥当性を確認することができる。
そこで、本稿では光ファイバー内での光の伝播モデルを構築し、光ファイバー内での光の伝播について解析的に評価することを試みる。

\section{光ファイバー内での光子の伝播モデル}
光ファイバーの中で光線は$\bm{x}_0 = (a, 0, 0)$で発光し、方向ベクトル$\bm{v}$で伝播し、半径$d$の円筒の$\bm{x}_1 = (x_1, y_1, z_1)$ でファイバー表面に衝突する(Figure \ref{fig:Single_cladding})。
光線の式は以下のように表される。
\begin{align}
  \bm{x} &= t \bm{v} + \bm{x}_0 = t \begin{pmatrix}
    \sin \theta \cos \phi \\
    \sin \theta \sin \phi \\
    \cos \theta
  \end{pmatrix} + \begin{pmatrix}
    a \\
    0 \\ 
    0
  \end{pmatrix}.
\end{align}
$x_1^2 + y_1^2 = d^2$を用いると、
\begin{align}
  x_1^2 + y_1^2 = (t \sin \theta \cos \phi + a)^2 + (t \sin \theta \sin \phi)^2 &= d^2 \nonumber \\
  (t \sin \theta)^2 + 2 a \cos \phi (t \sin \theta) + a^2 - d^2 &= 0 \nonumber \\
  t \sin \theta = -a \cos \phi + \sqrt{a^2 \cos^2 \phi - a^2 + d^2} &= -a \cos \phi + \sqrt{d^2 - a^2 \sin^2 \phi} \nonumber \\
  &\equiv -a \cos \phi + b, \; \; \;  b \equiv \sqrt{d^2 - a^2 \sin^2 \phi}. \label{eq:solt}
\end{align}
したがって、ファイバー表面に衝突する点$\bm{x}_1$の座標$(x_1, y_1)$は以下のように表される。
\begin{align}
  x_1 &= t \sin \theta \cos \phi + a = -a \cos^2 \phi + b \cos \phi  + a \label{eq:normx}\\
  y_1 &= t \sin \theta \sin \phi = -a \cos \phi \sin \phi + b \sin \phi. \label{eq:normy} 
\end{align}
$\bm{x}_1 = (x_1, y_1, z_1)$におけるファイバー表面での単位法線ベクトルは$(x_1, y_1, 0)$である。
\eqref{eq:normx}、\eqref{eq:normy}および$x_1^2 + y_1^2 = d^2$を満たすことから、$\bm{x}$での単位法線ベクトルの成分は$\bm{n}_1 = (x_1/d, y_1/d, 0)$となる。
\begin{figure}[h]
  \centering
  \includegraphics[width=0.9\columnwidth]{Figure/Single_cladding_fiber.png}
  \caption{Single-cladding ファイバー}
  \label{fig:Single_cladding}
\end{figure}

光のファイバー表面への入射角$\psi_1$(法線ベクトル$\bm{n}_1$に対する角度)は$\cos \psi_1 = \bm{n}_1 \cdot \bm{v}$を満たし、$\cos \psi_1$は次のように計算される。
\begin{align}
  \cos \psi_1 &= \frac{1}{d} \left( -a \sin \theta \cos^3 \phi + b \sin \theta \cos^2 \phi + a \sin \theta \cos \phi - a \sin \theta \cos \phi \sin^2 \phi + b \sin \theta \sin^2 \phi \right) \nonumber \\
  &= \frac{1}{d} \left( a \sin \theta \cos \phi (-\cos^2 \phi + 1 - \sin^2 \phi) + b \sin \theta \right) = \frac{b}{d} \sin \theta = \sin \theta \sqrt{1 - (a/d)^2 \sin^2 \phi}
 \label{eq:cospsi}
\end{align}
内部での全反射は、スネルの法則より入射角$\psi_1$が以下の関係を満たす時に起こる。
\begin{align}
  \sin \psi_1 &\ge \sin \psicone \equiv \frac{n_{c1}}{n_f} \label{eq:criticalangle1}
\end{align}
ここで、$\psicone$はファイバーのコアの屈折率$n_f$とクラッドの屈折率$n_{c1}$から決まる臨界角である。
全反射を起こす時に取り得る$\theta$の最大値$\theta_{\mathrm{max}}$は、$\psicone$と次のように対応する。
\begin{align}
  \cos \psicone &= \frac{b}{d} \sin \thetamax(\phi)  = \sin \thetamax(\phi) \sqrt{1 - (a/d)^2 \sin^2 \phi}.
\end{align}

光ファイバー内で光子が連続して2回反射する間の移動距離の差は次のように表される。
\begin{align}
\Delta \ell &= \left| t_+ - t_- \right| = \frac{2}{\sin \theta} b = \frac{2}{\sin \theta} \sqrt{d^2 - a^2 \sin^2 \phi}. \label{eq:deltaell} 
\end{align}

\section{コア軸からの距離$a$の関数としてのTrapping Efficiency}
\subsection{Single-cladding ファイバー}
まず、Trapping Efficiencyとは、光ファイバー内で発した光子のうち、全反射を起こしてファイバー内を伝播する光子の割合である。
光子が球面上に等方的に伝播していくと考えると、Trapping Efficiencyは全立体角のうち、ファイバー端に届く光子の立体角の割合を考えることで得られる。
このことより、半径$d$の光ファイバー内で、コア軸からの距離$a$で発した光子が式\eqref{eq:criticalangle1}を満たしながら全反射を起こす時のTrapping Efficiencyは式\eqref{eq:solidangle1}のように考えられる。
\begin{align}
  P(a) &= \frac{1}{4 \pi} \int_{\Phi} \int_0^{\thetamax(\phi, a)} \sin \theta \, d\theta \, d\phi = \frac{1}{4 \pi} \int_{\Phi} \int_{\mumin(\phi, a)}^1 \, d\mu \, d\phi \nonumber \\
  &= \frac{1}{4 \pi} \int_{\Phi} (1 - \mumin(\phi, a)) \, d\phi = \frac{1}{2} - \frac{1}{4\pi} \int_{\Phi} \mumin(\phi, a) \, d\phi,\label{eq:solidangle1}
\end{align}
ここで、$\mu \equiv \cos \theta$、$\mumin(\phi, a) \equiv \cos \theta_{\mathrm{max}}(\phi, a)$である。
$\mumin(\phi, a)$は式\eqref{eq:cospsic2}のように計算される。
\begin{align}
  \mumin(\phi, a) &\equiv \cos \theta_{\mathrm{max}}(\phi, a) = \left( 1 - \sin^2 \thetamax(\phi, a)\right)^{1/2} = \left(1  - \frac{\cos^2 \psicone}{1 - (a/d)^2 \sin^2 \phi} \right)^{1/2} \nonumber \\
  &= \left(1  - \frac{1 - (n_{c1}/n_f)^2}{1 - (a/d)^2 \sin^2 \phi} \right)^{1/2}.
  \label{eq:cospsic2}
\end{align}
式\eqref{eq:solidangle1}での$\phi$積分は、$\cos^2 \psicone/(1 - (a/d)^2 \sin^2 \phi) < 1$を満たすつまり、式\eqref{eq:cospsic2}を満たすような$\Phi$の範囲で行われる。
式\eqref{eq:solidangle1}の積分は数値積分により計算され、その結果が図\ref{fig:trapp_eff}である。
Trapping Efficiencyをコア軸からの距離$a$で平均化する計算は次の式で行われる。
\begin{align}
  \left<P\right> &= \frac{1}{\pi d^2} \int_0^{d} 2 \pi a P(a) \, da. \label{eq:Pd1}
\end{align}
Single-claddingファイバーの$n_0 = 1.59, n_1 = 1.49$と$d = 0.48$ mmに対して、$\left<P\right> = 0.061$と計算される。
これは図\ref{fig:trapp_eff}の水平な線と\ref{tab:my_label}の値を記されている。
\begin{figure}[h]
  \centering
  \includegraphics[width=0.9\columnwidth]{Figure/Trapping_Efficiency.png}
  \caption{single、double、iceは光子が全反射を起こす層を表している。緑と青、紫の線は式\eqref{eq:solidangle1}により計算されたTrapping Efficiency$P(a)$であり、赤と橙、茶色は式\eqref{eq:avezPa}で計算された、減衰長の効果を考慮に入れたTrapping Efficiency$P_{\mathrm{att}}(a)$である。}
  \label{fig:trapp_eff}
\end{figure}

\subsection{Double- と multi-cladding ファイバー}
\begin{figure}
  \centering
  \includegraphics[width=0.99\columnwidth]{Figure/Double_cladding_fiber.png}
  \caption{Double-cladding ファイバー}
  \label{fig:radii2}
\end{figure}
Double-claddingファイバーの場合、外側のcladの屈折率$n_2$を考慮する必要がある。
光が初期位置$\bm{x}_0 = (a, 0, 0)$から速度$\bm{v}(\theta, \phi)$で発せられた時を考える。
光は内側のcladに位置$\bm{x}_1$おいて、$\bm{n}_1$に対して入射角$\psi_1$で入射する。ここで、$\psi_1$は式\eqref{eq:cospsi}を満たす。
もし$\psi_1 \ge \psicone$ならば、$\bm{x}_1$において全反射が起こる。
しかし、$\psi_1 < \psicone$であるならば、光は内側のcladに屈折して侵入する。
外側のcladに対する入射角$\psi_2 (> \psi_1)$は、$n_f, n_{c1}$によってスネルの法則により決定され、次の式を満たす。
\begin{align}
  \frac{\sin \psi_2}{\sin \psi_1} &= \frac{n_f}{n_{c1}}. \label{eq:refcondition2}
\end{align}
この屈折により、光の進行方向$\bm{v}$は変化する。しかし、ここで変わるのは$\theta$のみであり、$\phi$は変化しない。
この屈折に関する式は次のように表される。
\begin{align}
  \sin \psi_2 &\ge \sin \psictwo \equiv \frac{n_{c2}}{n_{c1}}, \label{eq:psictwo}
\end{align}
ここで、$n_{c2}$は外側のcladの屈折率である。
式\eqref{eq:refcondition2}を用いると屈折に関する式は、$\psi_1$を用いて書き換えられる。
\begin{align}
  \sin \psi_1 &= \frac{n_{c1}}{n_f} \sin \psi_2 \ge \frac{n_{c1}}{n_f} \sin \psictwo = \frac{n_{c1}}{n_f} \frac{n_{c2}}{n_{c1}} = \frac{n_{c2}}{n_f} \label{eq:criticalangle2}
\end{align}
これより、multi-claddingファイバーに対しては全反射を起こす一番外側の層の屈折率$n_m$が屈折に関する式を決定することがわかる。
このことは次のように表される。
\begin{align}
  \sin \psi &> \frac{n_m}{n_f}. \label{eq:criticalangle3}
\end{align}
Trapping Efficiencyは式\eqref{eq:solidangle1}で与えられるが、式\eqref{eq:cospsic2}の代わりに$\mumin(\phi, a) = \cos \theta_{\mathrm{max}}(\phi, a)$を計算しておく必要がある。
\begin{align}
   \mumin(\phi, a) &= \cos \thetamax(\phi, a) = \left(1 - \sin^2 \thetamax(\phi, a) \right)^{1/2} = \left(1  - \frac{\cos^2 \psi_m}{1 - (a/d)^2 \sin^2 \phi} \right)^{1/2} \nonumber \\
  &= \left(1  - \frac{1 - (n_m/n_f)^2}{1 - (a/d)^2 \sin^2 \phi} \right)^{1/2},
  \label{eq:cospsic3}
\end{align}
ここで、$n_m$は最も外側のcladかファイバーを取り囲む周囲の物質(氷など)の屈折率であり、$\psi_m$はそれに対する入射角である。
$d_2$や$n_{c1}$などの新たな要素が現れるわけではないことに注意していただきたい。

\section{ファイバー端からの距離$z$でのTrapping Efficiency}


\section{ファイバー内での光子のさまざまなパラメータの分布}

\section{Fiber.hに実装されている関数の仕様}

\end{document}