\documentclass[dvipdfmx]{jlreq}
\usepackage{amsmath,amsfonts, amssymb}
\usepackage{bm}
\usepackage{physics}
\usepackage[dvipdfmx]{hyperref, graphicx}
\usepackage{tikz} 


\newcommand{\psicone}{\psi_{\mathrm{c1}}}
\newcommand{\psictwo}{\psi_{\mathrm{c2}}}
\newcommand{\psimin}{\psi_{\mathrm{min}}}
\newcommand{\psimax}{\psi_{\mathrm{max}}}
\newcommand{\thetamin}{\theta_{\mathrm{min}}}
\newcommand{\thetamax}{\theta_{\mathrm{max}}}
\newcommand{\phimin}{\phi_{\mathrm{min}}}
\newcommand{\phimax}{\phi_{\mathrm{max}}}
\newcommand{\Latt}{L_{\mathrm{att}}}
\newcommand{\mumin}{\mu_{\mathrm{min}}}

\hypersetup{
	colorlinks=blue, % リンクに色をつけない設定
	bookmarks=true, % 以下ブックマークに関する設定
	bookmarksnumbered=true,
	pdfborder={0 0 0},
	bookmarkstype=toc
}

\title{光ファイバー内で考えられる光の伝播モデル}
\date{\today}
\author{湯淺 圭太 \\ 大阪公立大学 理学研究科 宇宙線物理学研究室}
\begin{document}
\maketitle
\newpage
\tableofcontents
\newpage

\section{はじめに}
当研究室では、IceCube-Gen2で考えられるveto能力の低下を防ぐために、波長変換機能を持つ光ファイバーを用いた集光器の開発を行っている。
この集光器において、どれほどの光を集光できるかを評価するために、光ファイバーそのものの性能評価を行う必要がある。
そのためのシミュレーションとして、Geant4を用いた光ファイバー内での光の伝播シミュレーションを行っている。
このGeant4シミュレーション以外の手法を用いて、光ファイバー内での光の伝播を評価することができれば、Geant4のシミュレーション結果の妥当性を確認することができる。
そこで、本稿では光ファイバー内での光の伝播モデルを構築し、光ファイバー内での光の伝播について解析的に評価することを試みる。

\section{光ファイバー内での光子の伝播モデル}
光ファイバーの中で光線は$\bm{x}_0 = (a, 0, 0)$で発光し、方向ベクトル$\bm{v}$で伝播し、半径$d$の円筒の$\bm{x}_1 = (x_1, y_1, z_1)$ でファイバー表面に衝突する(Figure \ref{fig:Single_cladding})。
光線の式は以下のように表される。
\begin{align}
  \bm{x} &= t \bm{v} + \bm{x}_0 = t \begin{pmatrix}
    \sin \theta \cos \phi \\
    \sin \theta \sin \phi \\
    \cos \theta
  \end{pmatrix} + \begin{pmatrix}
    a \\
    0 \\ 
    0
  \end{pmatrix}.
\end{align}
$x_1^2 + y_1^2 = d^2$を用いると、
\begin{align}
  x_1^2 + y_1^2 = (t \sin \theta \cos \phi + a)^2 + (t \sin \theta \sin \phi)^2 &= d^2 \nonumber \\
  (t \sin \theta)^2 + 2 a \cos \phi (t \sin \theta) + a^2 - d^2 &= 0 \nonumber \\
  t \sin \theta = -a \cos \phi + \sqrt{a^2 \cos^2 \phi - a^2 + d^2} &= -a \cos \phi + \sqrt{d^2 - a^2 \sin^2 \phi} \nonumber \\
  &\equiv -a \cos \phi + b, \; \; \;  b \equiv \sqrt{d^2 - a^2 \sin^2 \phi}. \label{eq:solt}
\end{align}
したがって、ファイバー表面に衝突する点$\bm{x}_1$の座標$(x_1, y_1)$は以下のように表される。
\begin{align}
  x_1 &= t \sin \theta \cos \phi + a = -a \cos^2 \phi + b \cos \phi  + a \label{eq:normx}\\
  y_1 &= t \sin \theta \sin \phi = -a \cos \phi \sin \phi + b \sin \phi. \label{eq:normy} 
\end{align}
$\bm{x}_1 = (x_1, y_1, z_1)$におけるファイバー表面での単位法線ベクトルは$(x_1, y_1, 0)$である。
\eqref{eq:normx}、\eqref{eq:normy}および$x_1^2 + y_1^2 = d^2$を満たすことから、$\bm{x}$での単位法線ベクトルの成分は$\bm{n}_1 = (x_1/d, y_1/d, 0)$となる。
\begin{figure}[h]
  \centering
  \includegraphics[width=0.9\columnwidth]{Figure/Single_cladding_fiber.png}
  \caption{Single-cladding ファイバー}
  \label{fig:Single_cladding}
\end{figure}

光のファイバー表面への入射角$\psi_1$(法線ベクトル$\bm{n}_1$に対する角度)は$\cos \psi_1 = \bm{n}_1 \cdot \bm{v}$を満たし、$\cos \psi_1$は次のように計算される。
\begin{align}
  \cos \psi_1 &= \frac{1}{d} \left( -a \sin \theta \cos^3 \phi + b \sin \theta \cos^2 \phi + a \sin \theta \cos \phi - a \sin \theta \cos \phi \sin^2 \phi + b \sin \theta \sin^2 \phi \right) \nonumber \\
  &= \frac{1}{d} \left( a \sin \theta \cos \phi (-\cos^2 \phi + 1 - \sin^2 \phi) + b \sin \theta \right) = \frac{b}{d} \sin \theta = \sin \theta \sqrt{1 - (a/d)^2 \sin^2 \phi}
 \label{eq:cospsi}
\end{align}
内部での全反射は、スネルの法則より入射角$\psi_1$が以下の関係を満たす時に起こる。
\begin{align}
  \sin \psi_1 &\ge \sin \psicone \equiv \frac{n_{c1}}{n_f} \label{eq:criticalangle1}
\end{align}
ここで、$\psicone$はファイバーのコアの屈折率$n_f$とクラッドの屈折率$n_{c1}$から決まる臨界角である。
全反射を起こす時に取り得る$\theta$の最大値$\theta_{\mathrm{max}}$は、$\psicone$と次のように対応する。
\begin{align}
  \cos \psicone &= \frac{b}{d} \sin \thetamax(\phi)  = \sin \thetamax(\phi) \sqrt{1 - (a/d)^2 \sin^2 \phi}. \label{eq:reflection_max}
\end{align}

光ファイバー内で光子が連続して2回反射する間の移動距離の差は次のように表される。
\begin{align}
\Delta \ell &= \left| t_+ - t_- \right| = \frac{2}{\sin \theta} b = \frac{2}{\sin \theta} \sqrt{d^2 - a^2 \sin^2 \phi}. \label{eq:deltaell} 
\end{align}

\section{ファイバー内で発光する初期位置$a$}
\begin{figure}[h]
  \centering
  \includegraphics[width=0.7\columnwidth]{Figure/Distribution_of_a.png}
  \caption{Distribution of Initial Position}
  \label{fg:Distribution_of_Initial_Position}
\end{figure}
実際のファイバーでは、外部から入射してきた光を吸収し、ファイバー内で発光することで光を伝えている。
この、光が吸収されるまでの距離を吸収長とよび、入射した光の光子数が$1/e$になるまでの距離である。
この吸収の効果のために、ファイバー内で発光する地点は一様ではない。
そこで、ある吸収長に対する発光位置のコア軸からの距離$a$の分布考えることにより、擬似的にファイバーの外から光が入射した状況を考える。
これに関する詳細は、古前壱朗朗氏によって記述された別資料に記載されている。
その資料によると、ファイバーに対して垂直に入射した光子が吸収される地点$a$の分布関数は次のようになる。
\begin{equation}
  \frac{dN}{da} = \frac{2}{l_{abs}} \exp(\frac{-r \cos\psi}{l_{abs}}) \cosh(\frac{\sqrt{a^2-r^2\sin^2\psi}}{l_{abs}}) \frac{a}{\sqrt{a^2-r^2\sin^2\psi}}
  \label{eq:Initial_Position}
\end{equation}
この結果は規格化されたものである。また、$r = 半径、l_{abs} = 吸収長, \psi = ファイバーに入射する光子の角度$である。
この結果を$\psi$について平均化すると次のようになる。
ここで具体的な結果が書けないのは、前述の式\ref{eq:Initial_Position}の積分は解析的に行えないためである。
\begin{equation}
  \left<\frac{dN}{da}\right>_{\psi} = \int_{0}^{\frac{\pi}{2}} \frac{dN}{da} d\psi
\end{equation}
これの$a$分布を表したものが図\ref{fg:Distribution_of_Initial_Position}である。
以下aで積分を行う際に、この分布を重み付として積分を行うことで吸収の効果を組み込んでいる。
もし$a$が一様に分布する場合を考えたいのであれば、$\left<\frac{dN}{da}\right>_{\psi}=a$と置き換えて計算すれば良い。

\section{コア軸からの距離$a$の関数としてのTrapping Efficiency}
\subsection{Single-cladding ファイバー}
まず、Trapping Efficiencyとは、光ファイバー内で発した光子のうち、全反射を起こしてファイバー内を伝播する光子の割合である。
光子が球面上に等方的に伝播していくと考えると、Trapping Efficiencyは全立体角のうち、ファイバー端に届く光子の立体角の割合を考えることで得られる。
このことより、半径$d$の光ファイバー内で、コア軸からの距離$a$で発した光子が式\eqref{eq:criticalangle1}を満たしながら全反射を起こす時のTrapping Efficiencyは式\eqref{eq:solidangle1}のように考えられる。
\begin{align}
  P(a) &= \frac{1}{4 \pi} \int_{\Phi} \int_0^{\thetamax(\phi, a)} \sin \theta \, d\theta \, d\phi = \frac{1}{4 \pi} \int_{\Phi} \int_{\mumin(\phi, a)}^1 \, d\mu \, d\phi \nonumber \\
  &= \frac{1}{4 \pi} \int_{\Phi} (1 - \mumin(\phi, a)) \, d\phi = \frac{1}{2} - \frac{1}{4\pi} \int_{\Phi} \mumin(\phi, a) \, d\phi,\label{eq:solidangle1}
\end{align}
ここで、$\mu \equiv \cos \theta$、$\mumin(\phi, a) \equiv \cos \theta_{\mathrm{max}}(\phi, a)$である。
$\mumin(\phi, a)$は式\eqref{eq:cospsic2}のように計算される。
\begin{align}
  \mumin(\phi, a) &\equiv \cos \theta_{\mathrm{max}}(\phi, a) = \left( 1 - \sin^2 \thetamax(\phi, a)\right)^{1/2} = \left(1  - \frac{\cos^2 \psicone}{1 - (a/d)^2 \sin^2 \phi} \right)^{1/2} \nonumber \\
  &= \left(1  - \frac{1 - (n_{c1}/n_f)^2}{1 - (a/d)^2 \sin^2 \phi} \right)^{1/2}.
  \label{eq:cospsic2}
\end{align}
式\eqref{eq:solidangle1}での$\phi$積分は、$\cos^2 \psicone/(1 - (a/d)^2 \sin^2 \phi) < 1$を満たすつまり、式\eqref{eq:cospsic2}を満たすような$\Phi$の範囲で行われる。
式\eqref{eq:solidangle1}の積分は数値積分により計算され、その結果が図\ref{fig:trapp_eff}である。
Trapping Efficiencyをコア軸からの距離$a$で平均化する計算は次の式で行われる。
\begin{align}
  \left<P\right> &= \frac{1}{\int_0^{d} 2 \pi \left<\frac{dN}{da}\right>_{\psi} \, da} \int_0^{d} 2 \pi \left<\frac{dN}{da}\right>_{\psi} P(a) \, da. \label{eq:Pd1}
\end{align}
Single-claddingファイバーの$n_0 = 1.59, n_1 = 1.49$と$d = 0.48$ mmに対して、$\left<P\right> = 0.061$と計算される。
これは図\ref{fig:trapp_eff}の水平な線と\ref{tab:my_label}の値を記されている。
\begin{figure}[h]
  \centering
  \includegraphics[width=0.9\columnwidth]{Figure/Trapping_Efficiency.png}
  \caption{single、double、iceは光子が全反射を起こす層を表している。緑と青、紫の線は式\eqref{eq:solidangle1}により計算されたTrapping Efficiency$P(a)$であり、赤と橙、茶色は式\eqref{eq:avezPa}で計算された、減衰長の効果を考慮に入れたTrapping Efficiency$P_{\mathrm{att}}(a)$である。}
  \label{fig:trapp_eff}
\end{figure}

\subsection{Double- と multi-cladding ファイバー}
\begin{figure}
  \centering
  \includegraphics[width=0.99\columnwidth]{Figure/Double_cladding_fiber.png}
  \caption{Double-cladding ファイバー}
  \label{fig:radii2}
\end{figure}
Double-claddingファイバーの場合、外側のcladの屈折率$n_2$を考慮する必要がある。
光が初期位置$\bm{x}_0 = (a, 0, 0)$から速度$\bm{v}(\theta, \phi)$で発せられた時を考える。
光は内側のcladに位置$\bm{x}_1$おいて、$\bm{n}_1$に対して入射角$\psi_1$で入射する。ここで、$\psi_1$は式\eqref{eq:cospsi}を満たす。
もし$\psi_1 \ge \psicone$ならば、$\bm{x}_1$において全反射が起こる。
しかし、$\psi_1 < \psicone$であるならば、光は内側のcladに屈折して侵入する。
外側のcladに対する入射角$\psi_2 (> \psi_1)$は、$n_f, n_{c1}$によってスネルの法則により決定され、次の式を満たす。
\begin{align}
  \frac{\sin \psi_2}{\sin \psi_1} &= \frac{n_f}{n_{c1}}. \label{eq:refcondition2}
\end{align}
この屈折により、光の進行方向$\bm{v}$は変化する。しかし、ここで変わるのは$\theta$のみであり、$\phi$は変化しない。
この屈折に関する式は次のように表される。
\begin{align}
  \sin \psi_2 &\ge \sin \psictwo \equiv \frac{n_{c2}}{n_{c1}}, \label{eq:psictwo}
\end{align}
ここで、$n_{c2}$は外側のcladの屈折率である。
式\eqref{eq:refcondition2}を用いると屈折に関する式は、$\psi_1$を用いて書き換えられる。
\begin{align}
  \sin \psi_1 &= \frac{n_{c1}}{n_f} \sin \psi_2 \ge \frac{n_{c1}}{n_f} \sin \psictwo = \frac{n_{c1}}{n_f} \frac{n_{c2}}{n_{c1}} = \frac{n_{c2}}{n_f} \label{eq:criticalangle2}
\end{align}
これより、multi-claddingファイバーに対しては全反射を起こす一番外側の層の屈折率$n_m$が屈折に関する式を決定することがわかる。
このことは次のように表される。
\begin{align}
  \sin \psi &> \frac{n_m}{n_f}. \label{eq:criticalangle3}
\end{align}
Trapping Efficiencyは式\eqref{eq:solidangle1}で与えられるが、式\eqref{eq:cospsic2}の代わりに$\mumin(\phi, a) = \cos \theta_{\mathrm{max}}(\phi, a)$を計算しておく必要がある。
\begin{align}
   \mumin(\phi, a) &= \cos \thetamax(\phi, a) = \left(1 - \sin^2 \thetamax(\phi, a) \right)^{1/2} = \left(1  - \frac{\cos^2 \psi_m}{1 - (a/d)^2 \sin^2 \phi} \right)^{1/2} \nonumber \\
  &= \left(1  - \frac{1 - (n_m/n_f)^2}{1 - (a/d)^2 \sin^2 \phi} \right)^{1/2},
  \label{eq:cospsic3}
\end{align}
ここで、$n_m$は最も外側のcladかファイバーを取り囲む周囲の物質(氷など)の屈折率であり、$\psi_m$はそれに対する入射角である。
$d_2$や$n_{c1}$などの新たな要素が現れるわけではないことに注意していただきたい。

\section{ファイバー端からの距離$z$でのTrapping Efficiency}
光はファイバー中で減衰し、元の光子数の$1/e$の数に減衰するのに必要な長さを減衰長という。
この減衰の効果を考慮に入れるとTrapping Efficiencyの計算は式\eqref{eq:solidangle1}から次のように書き換えられる。
\begin{align}
  P(a, z) &= \frac{1}{4 \pi} \int_{\Phi} \int_{\mumin(\phi, a)}^1 \exp\left( -\frac{L - z}{\mu \Latt}\right) \, d\mu \, d\phi, \label{eq:Paz}
\end{align}
ここで$z$はファイバー内で光が発せらるファイバー端からの位置であり、$(L-z)/\cos \theta = (L-z)/\mu$は光の実質的な伝播距離、$\exp(-(L-z)/\mu \Latt)$は光の減衰の効果を表している。
光が長さ$L$のファイバー内で一様に発せられると考えた場合、$z$で平均化されたTrapping Efficiencyは次のようになる。
\begin{align}
  P_{\mathrm{att}}(a) &= \frac{1}{L} \int_0^L P(a, z) \, dz \label{eq:avezPa}
\end{align}
この計算結果は、図\ref{fig:trapp_eff}に示されている。

光が体積$V = \pi d^2 L$の中で一様かつ等方的に発せられた場合、Trapping Efficiencyは次のように計算される。
\begin{align}
  \left<P_{\mathrm{att}}\right> &= \frac{1}{V} \int_V P(a, z) \, dV = \frac{1}{\int_0^{d} 2 \pi \left<\frac{dN}{da}\right>_{\psi} \, da} \int_0^d 2 \pi P_{\mathrm{att}}(a) \left<\frac{dN}{da}\right>_{\psi} \, da \label{eq:P-aveaz}
\end{align}

\begin{table}
  \centering
  \caption{長さ$L = 1000$ mm、$\Latt = 4000$ mmのsingle- と double-claddingのファイバーに対するTrapping Efficienciesの一覧}
  \begin{tabular}{cccccc}
  Cladding  & $n_0$  & $n_1$ & $n_2$ & $\left<P\right>$ & $\left<P_{\mathrm{att}}\right>$  \\ \hline
Single & $1.59$ & $1.49$ & -   & $0.06092 \pm 0.00000$ & $0.05154 \pm 0.00002 $ \\
Double & $1.59$ & $1.49$ & $1.42$ & $0.10120 \pm 0.00000$ & $0.08438 \pm 0.00001$
  \end{tabular}
  \label{tab:my_label}
\end{table}

\section{ファイバー内での光子のさまざまなパラメータの分布}
\subsection{角度分布}
理想的なファイバーにおいて、ファイバーの軸に対する角度$\theta$は伝播する光子に対して保存される。
つまり、cladやファイバーの外側の物質との境界で起きる全反射では、角度$\theta$は変化しない。
これは、clad内での全反射において、光は全反射の前と後で計2度屈折し、その影響は打ち消しあうためである。
その結果として、初期の$\theta$の値は保存される。
このことより、伝播する光が方位角$\phi$に対して一様に発せられる時、ファイバー内を伝播する光の$\theta$分布は次のように表される。
\begin{align}
  \frac{dP(a, \theta)}{d\theta} &= \frac{1}{2 \pi} \sin \theta \int_{\Phi(\theta)}  \, d \phi, \label{eq:dPdtheta}
\end{align}
ここで、任意の$a$と$\theta$に対して、$\phi$に関する積分の区間は、式\eqref{eq:reflection_max}を$\phi$に適用して得られる$\Phi(\theta)$の範囲に制限される。
この積分の項が、任意の$a$と$\theta$に対して光が全反射できるかどうかを実質的に表すものになる。

\begin{figure}
  \centering
  \includegraphics[scale=0.6]{Figure/Trapping_Efficiency_Theta.png}
  \caption{$\theta$ distribution of photons at $z = L$.}
  \label{fig:Trapping_Efficiency_Theta}
\end{figure}

体積$V = \pi d^2 L$のファイバー内で一様かつ等方的に発せられる光に対して、微小体積$dV = 2 \pi a \, da \, dz$で平均化すると、ファイバー端における平均角度分布は次のように表される。
\begin{align}
  \left< \frac{dP(\theta)}{d\theta} \right> &= \frac{1}{V} \int_V \frac{dP(a, \theta)}{d\theta} \exp \left( -\frac{L - z}{\cos \theta \Latt} \right) \left<\frac{dN}{da}\right>_{\psi} \, dV \nonumber \\
  &= A(\theta) \int_0^d 2 \pi \frac{dP(a, \theta)}{d\theta}  \left<\frac{dN}{da}\right>_{\psi} \, da. \label{eq:dPdtheta-aveV}\\
  A(\theta) &\equiv \frac{\cos \theta \Latt}{V} \left( 1 - \exp \left( -\frac{L}{\cos \theta \Latt} \right)\right)  \nonumber
\end{align}
この計算結果は図\ref{fig:Trapping_Efficiency_Theta}に示されている。

\subsection{移動距離分布}
角度分布の式\eqref{eq:dPdtheta}と式\eqref{eq:dPdtheta-aveV}は次のように表すことができる。
\begin{align}
   \frac{dP(a, \theta)}{d \cos \theta} &= \frac{dP(a, \theta)}{d \theta} \frac{d \theta}{d \cos \theta} = -\frac{1}{\sin \theta}\frac{dP(a, \theta)}{d \theta} = -\frac{1}{2 \pi} \int_{\Phi(\theta)}  \, d \phi, \label{eq:dPdcostheta} \\
     \left< \frac{dP(\theta)}{d\cos\theta} \right> &= \frac{1}{V} \int_V \frac{dP(a, \theta)}{d\cos\theta} \exp \left( -\frac{L - z}{\cos \theta \Latt} \right) \left<\frac{dN}{da}\right>_{\psi} \, dV = A(\theta) \int_0^d 2 \pi \frac{dP(a, \theta)}{d\cos\theta} \left<\frac{dN}{da}\right>_{\psi} \, da \label{eq:dPdcostheta-aveV}
\end{align}
ここで、$\cos\theta$は光の移動距離$\ell(z, \, \theta)$へと次の式のように対応することができる。
\begin{align}
\ell(z, \, \theta) = \frac{L-z}{\cos \theta}, \label{eq:pathlengthcostheta}
\end{align}
これより、
\begin{align*}
  \frac{d\cos\theta}{d\ell} &= - \frac{L-z}{\ell^2}
\end{align*}
したがってTrapping Efficiencyの移動距離分布は次のように記述される。
\begin{align}
  \frac{dP(a, z)}{dl} = \frac{dP(a, z)}{d\cos\theta} \frac{d\cos\theta}{dl} = - \frac{L-z}{l^2} \frac{dP(a, z)}{d\cos\theta} = \frac{1}{2 \pi} \frac{L-z}{l^2} \int_{\Phi(\theta)} \label{eq:path_length_dist}
\end{align}
同様にしてTrapping Efficiencyの移動距離をコア軸からの距離$a$とファイバー端からの距離$z$で平均化したものを考えると
\begin{align}
   \left< \frac{dP(\theta)}{d\ell} \right> &= \frac{1}{V} \int_V \frac{dP(a, \theta)}{d\cos\theta} \frac{d\cos\theta}{d\ell} \exp \left( -\frac{L - z}{\cos \theta \Latt} \right) \left<\frac{dN}{da}\right>_{\psi}\, dV \nonumber \\
   &= \frac{1}{V} \exp \left( -\frac{\ell}{\Latt} \right)  \int_0^L \frac{z-L}{\ell^2} \,  dz \int_0^d 2 \pi \frac{dP(a, \theta)}{d\cos\theta} \left<\frac{dN}{da}\right>_{\psi} \, da \nonumber \\
   &= B(\theta) \int_0^d 2 \pi a  \frac{dP(a, \theta)}{d\cos\theta} \left<\frac{dN}{da}\right>_{\psi} \, da
  \label{eq:dPdell-aveV}\\
  B(\theta) &\equiv  \frac{1}{V} \frac{\sin \theta \Latt}{\cos \theta}  \left( \cos \theta \Latt - e^{-L/\cos \theta \Latt} (\cos \theta \Latt + L) \right) \nonumber
\end{align}

\subsection{光がファイバー内で移動する時間の分布}
式\eqref{eq:path_length_dist}より、$\frac{dl}{dt} = c'$は明らかなので、Trapping Efficiencyの光の移動時間分布は次のように計算される。
\begin{equation}
  \frac{dP(a, z)}{dt} =\frac{dP(a, z)}{dl} \frac{dl}{dt} = \frac{c' \cos^2 \theta}{L-z} \frac{dP(a, z)}{d\cos \theta} = \frac{c'}{2 \pi} \frac{\cos^2\theta}{L-z}\int_{\Phi(\theta)} d \phi \label{eq:dPdt}
\end{equation}
これの$\theta $に関する平均化を行う。また、attenuationの効果を考慮に入れるため、減衰項を含める。
\begin{align}
  \left< \frac{dP}{dt} \right>_\theta &= \int_{0}^{\pi/2}\sin \theta \frac{dP(a, z)} \exp\left( -\frac{L - z}{\cos \theta \Latt} \right){dt} d\theta \nonumber\\
  &= \frac{c'}{2 \pi} \int_{0}^{\pi/2} \int_{\Phi(\theta)} \sin \theta \frac{\cos^2\theta}{L-z}\exp\left( -\frac{L - z}{\cos \theta \Latt}\right) d \phi d\theta \nonumber\\
  &= c' \int_{0}^{\pi/2} \frac{\cos^2\theta}{L-z} \exp\left( -\frac{L - z}{\cos \theta \Latt}\right) \frac{dP(a, \theta)}{d\theta} d\theta 
  \label{eq:averaged_dPdt}
\end{align}
\eqref{eq:averaged_dPdt}を用いて、それぞれの変数を細かなスッテプで刻むことにより、時間分布を求めることができる。
それが以下である。
\begin{figure}[h]
  \centering
  \begin{minipage}[b]{0.49\columnwidth}
    \centering
    \includegraphics[width=0.9\columnwidth]{Figure/Trapping_Efficiency_with_Absorption_Lenfth_log.png}
    \caption{$\theta$ Transit Time Distribution of Trapping Efficiency (log scale)}
    \label{fig:dPdt_log}  
  \end{minipage}
  \begin{minipage}[b]{0.49\columnwidth}
    \centering
    \includegraphics[width=0.9\columnwidth]{Figure/Trapping_Efficiency_with_Absorption_Lenfth_loglog.png}
    \caption{$\theta$ Transit Time Distribution of Trapping Efficiency (loglog scale)}
    \label{fig:dPdt_loglog}      
  \end{minipage}
\end{figure}

このグラフの横軸はTransit Timeであり、縦軸はTrapping Efficiencyの時間分布、色の濃淡はz分布を表す。
また、Efficiencyが大きいものはCore中心から遠く、Efficiencyが小さいものはCore中心に近い。

次に、5章で論じた吸収長の効果を考慮に入れる。
式\eqref{eq:Initial_Position}と式\eqref{eq:dPdt}より、$a$の初期位置の分布を重みとして平均化したTrapping Efficiencyの時間分布は次のようになる。
\begin{equation}
  \left<\frac{dP}{dt}\right>_a = \int_{0}^{d}\left< \frac{dP}{dt} \right>_\theta \times \left<\frac{dN}{da}\right>_{\psi}da
  \label{eq:dPdt_with_a_dist}
\end{equation}
Geant4の計算では$z$を固定しているため、$z$を固定して計算をすると図\ref{fig:dPdt_ave_a_and_theta}のようになる。
\begin{figure}[h]
  \centering
  \includegraphics[width=0.9\columnwidth]{Figure/Trapping_Efficiency_with_Absorption_averaged_a_and_theta.png}
  \caption{Trapping Efficiency Time Distribution with Initial Position Distribution}
  \label{fig:dPdt_ave_a_and_theta}
\end{figure}

最後に、式\eqref{eq:dPdt_with_a_dist}を$z$についても平均化する。
\begin{equation}
  \left<\frac{dP}{dt}\right> = \frac{1}{L} \int_0^L \left<\frac{dP}{dt}\right>_a dz
  \label{eq:dPdt_ave}
\end{equation}

\subsection{Trapping Efficiencyのsec分布}
\eqref{eq:pathlengthcostheta}より、光の電波時間を考えると次の式のように求まる。
\begin{equation}
  t = \frac{l}{c'} = \frac{1}{c'}\frac{L-z}{\cos\theta}
\end{equation}
ここで$c'$はファイバー中での高速である。
このことより、伝達時間と$\theta$には次のような関係がある。
\begin{equation}
  t = \frac{1}{\cos\theta}
\end{equation}
よって、Trapping Efficiencyの時間分布はその$\cos^{-1}(\theta) = \sec(\theta)$分布と似たような形になると思われる。
そこで、Trapping Efficiencyのsec分布を考える。式\eqref{eq:dPdtheta-aveV}より、secの分布は次の式に従う。
\begin{equation}
  \frac{dP}{d\frac{1}{\cos\theta}} = \frac{\cos^2\theta}{\sin\theta}\frac{dP}{d\theta} = \frac{\cos^2\theta}{\sin\theta}A(\theta) \int_0^d 2 \pi \frac{dP(a, \theta)}{d\theta} \left<\frac{dN}{da}\right>_{\psi} da
  \label{dPdsec}
\end{equation}
これを$\theta$で平均化すると
\begin{equation}
  \left<\frac{dP}{d\frac{1}{\cos\theta}}\right>_{\theta} = \cos^2 \theta A(\theta) \int_0^d 2 \pi \frac{dP(a, \theta)}{d\theta} \left<\frac{dN}{da}\right>_{\psi} da
  \label{dPdsec_ave}
\end{equation}
この$\sec(\theta)$分布(図\ref{fig:dPdSec_theta})が$<dP/dt>_a$(図\ref{fig:dPdt_ave_a_and_theta})と似た形になることがわかる。
グラフの$x$軸の最大位置を揃え、重ね合わせたグラフが図\ref{fig:dPdSec_dPdt}である。

\begin{figure}[h]
  \centering
  \includegraphics[width=0.9\columnwidth]{Figure/dPdSec_theta.png}
  \caption{Trapping Efficiency Second Theta Distribution}
  \label{fig:dPdSec_theta}
\end{figure}
\begin{figure}[h]
  \centering
  \includegraphics[width=0.9\columnwidth]{Figure/dPdt_and_dPdSec_same_size.png}
  \caption{Overlay of Trapping Efficiency Second Theta Distribution and Time Distribution}
  \label{fig:dPdSec_dPdt}
\end{figure}

\subsection{脱出角度分布}
脱出角度はTrapping Efficiencyのコア軸からの角度$\theta$によってスネルの法則により決定される。
コア軸からの角度$\theta$をファイバー端への入射角と考え、脱出角度を$\theta_{esc}$とすると、スネルの法則より
\begin{equation}
  \sin \theta_{esc} = \frac{n_core}{n_f} \sin \theta
\end{equation}
ここで$n_{core}$はファイバーコアの屈折率、$n_f$はファイバー外の物質の屈折率である。
以上より、Trapping Efficiencyの脱出角度分布は次の式で求められる。
\begin{equation}
  \frac{dP}{d\theta{core}} = \frac{dP}{d\sin\theta_{esc}}\frac{d\sin\theta_{esc}}{d\theta_{esc}} = \sqrt{1-sin^2\theta_{esc}}\frac{dP}{\frac{n_{core}}{n_f}d\sin\theta} = \frac{n_f}{n_{core}}\frac{\sqrt{1-sin^2\theta_{esc}}}{\cos\theta}\frac{dP}{d\theta}
\end{equation}
これにAbsorptionの効果を重み付として平均化したものが図\ref{fig:Escape_Angle_Distribution}である。なお、ここでは$n_f=1.309$(氷)として考えている。
\begin{figure}
  \centering
  \includegraphics[width=0.9\columnwidth]{Figure/Escape_Angle_Distribution_peak.png}
  \caption{Trapping Efficiency Escape Angle Distribution}
  \label{fig:Escape_Angle_Distribution}
\end{figure}

\section{反射率によるTrapping Efficiencyの変化}
ファイバー内で発行した光は、理想的な状態においては、最外層のcladとの境界で全反射を起こし続ける。
しかし、実際には境界での反射率は100\%ではなく、いくらかの光が透過してファイバー外に逃げてしまう。
そこで、境界での反射率$R$を考慮に入れたTrapping Efficiencyを計算する。
光がコア軸からの距離$a$、方位角$\phi$で放射されたとする。
光がファイバー内で一度反射し、次に反射するまでの距離を$l_{step}$とすると、$l_{step}$のファイバー断面に射影した時の長さ$l_{step}^{xy}$は次の式で表される。
\begin{align}
  l_{step}^{xy} &= 2 \sqrt{d^2-(a \sin \phi)^2} \label{eq:Lstep}
\end{align}
よって、$l_{step}$は次のように表される。
\begin{align}
  l_{step} &= \frac{l_{step}^{xy}}{\sin \theta} = \frac{2 \sqrt{d^2-(a \sin \phi)^2}}{\sin \theta} \label{eq:Lstep_full}
\end{align}
光がファイバー内で伝播する距離$l$に対して、反射回数$N_{ref}$は次のように表される。
\begin{align}
  N_{ref} &= \frac{l}{l_{step}} = \frac{l \sin \theta}{2 \sqrt{d^2-(a \sin \phi)^2}}
\end{align}
ここで、式(\eqref{eq:pathlengthcostheta})より$l = (L-z)/\cos \theta$であることに注意すると、反射回数は次のように表される。
\begin{align}
  N_{ref} = \frac{(L-z)/\cos \theta \cdot \sin \theta}{2 \sqrt{d^2-(a \sin \phi)^2}} &= \frac{(L-z) \tan \theta}{2 \sqrt{d^2-(a \sin \phi)^2}} \nonumber \\
  &= \frac{(L-z)}{2 \sqrt{d^2-(a \sin \phi)^2}}\sqrt{\frac{1}{\cos^2 \theta} - 1} \nonumber\\ &= \frac{(L-z)}{2 \sqrt{d^2-(a \sin \phi)^2}}\sqrt{\frac{1}{\mu^2} - 1}
  \label{eq:Nref}
\end{align}
ここで、式\eqref{eq:Paz}と同様にすると、反射率$R$を考慮に入れたTrapping Efficiencyは次のように表される。
\begin{align}
  P_R(a, z) &= \frac{1}{4 \pi} \int_{\Phi} \int_{\mumin(\phi, a)}^1 R^{N_{ref}} \exp\left( -\frac{L - z}{\mu \Latt}\right) \, d\mu \, d\phi, \label{eq:PRaz}
\end{align}
また、光がファイバー内で一様に発せられると考えた場合、式\eqref{eq:avezPa}と同様にして$z$で平均化されたTrapping Efficiencyは次のようになる。
\begin{align}
  P_{R, \mathrm{att}}(a) &= \frac{1}{L} \int_0^L P_R(a, z) \, dz \label{eq:PRatt}
\end{align}
よって、反射率を考慮に入れた場合のTrapping Efficiencyは次のように計算される。
\begin{align}
  \left<P_R\right> &= \frac{1}{V}\int_{V} P_R(a, z) \, dV =
  \frac{1}{\int_0^{d} 2 \pi \left<\frac{dN}{da}\right>_{\psi} \, da} \int_{0}^{d} R^{N_{ref}} 2 \pi P_{R, \mathrm{att}}(a)\left<\frac{dN}{da}\right>_{\psi}\, da \label{eq:PRaz}
\end{align}
ここで、$N_{ref}$は式\eqref{eq:Nref}で与えられる。

\section{Fiber.hに実装されている関数の仕様}

\end{document}